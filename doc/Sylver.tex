\documentclass{documentation}

\title{Sylver Package}

\author{Joshua Maglione}
\address{Universit\"at Bielefeld}
\email{jmaglione@math.uni-bielefeld.de}

\author{James B. Wilson}
\address{Colorado State University}
\email{james.wilson@colostate.edu}

\version{1.0}
\date{\today}
\copyrightyear{2016--2019}


\usepackage{minitoc}
\usepackage{stmaryrd}  % \llbracket
\usepackage{wasysym}  %\Leftcircle


\newcommand{\midarrow}{\tikz \draw[-triangle 90] (0,0) -- +(.1,0);}
\newcommand{\midArrow}{\tikz \draw[-triangle 90] (0,0) -- +(.1,0);
					   \tikz \draw[-triangle 90] (.1,0) -- +(.1,0);}


\newcommand*{\udots}{\reflectbox{$\ddots$}}

%\makeatletter
%\providecommand*{\Dashv}{%
%  \mathrel{%
%    \mathpalette\@Dashv\vDash
%  }%
%}
%\newcommand*{\@Dashv}[2]{%
%  \reflectbox{$\m@th#1#2$}%
%}
%\makeatother

\makeatletter
\DeclareRobustCommand*\cal{\@fontswitch\relax\mathcal}
\makeatother

\makeatletter
\DeclareRobustCommand*\frak{\@fontswitch\relax\mathfrak}
\makeatother

\newcommand{\onto}{\twoheadrightarrow}

%\DeclareMathAlphabet{\mathpzc}{OT1}{pzc}{m}{it}

% Create chapter autors
\newcommand\chapterauthor[2]{\authortoc{#1}{#2}\printchapterauthor{#1}{#2}}
\makeatletter
\newcommand{\printchapterauthor}[2]{%
  {\parindent0pt\vspace*{-25pt}%
  \linespread{1.1}\large\scshape#1 \\ \footnotesize#2%
  \par\nobreak\vspace*{35pt}}
  \@afterheading%
}
\newcommand{\authortoc}[2]{%
  \addtocontents{toc}{\vskip-10pt}%
  \addtocontents{toc}{%
    \protect\contentsline{chapter}%
    {\hskip1em\mdseries\scshape\protect\scriptsize#1\quad \tiny#2}{}{}}
  \addtocontents{toc}{\vskip5pt}%
}
\makeatother


%-----------------------------------------------------
%       Standard theoremlike environments.
%-----------------------------------------------------
%% \theoremstyle{plain} %% This is the default
\numberwithin{equation}{section}
\newtheorem{thm}{Theorem}
\newtheorem*{thm*}{Theorem}
\newtheorem{mainthm}{Theorem}
\renewcommand*{\themainthm}{\Alph{mainthm}}
\newtheorem{lem}[equation]{Lemma}
\newtheorem{prop}[equation]{Proposition}
\newtheorem{prob}[equation]{Problem}
\newtheorem{lemma}[equation]{Lemma}
\newtheorem{ex}[equation]{Example}
\newtheorem{coro}[equation]{Corollary}
\newtheorem*{coro*}{Corollary}

\theoremstyle{remark}
\newtheorem*{remark*}{Remark}
\newtheorem{remark}[equation]{Remark}

\theoremstyle{definition}
\newtheorem{defn}[equation]{Definition}

\numberwithin{figure}{section}
\numberwithin{table}{section}

%--------------------------------------------------
%       Item references.
%-------------------------------------------------- 
\newcommand{\exref}[1]{Ex\-am\-ple \ref{#1}}
\newcommand{\thmref}[1]{Theo\-rem \ref{#1}}
\newcommand{\defref}[1]{Def\-i\-ni\-tion \ref{#1}}
\newcommand{\eqnref}[1]{(\ref{#1})}
\newcommand{\secref}[1]{Sec\-tion \ref{#1}}
\newcommand{\lemref}[1]{Lem\-ma \ref{#1}}
\newcommand{\propref}[1]{Prop\-o\-si\-tion \ref{#1}}
\newcommand{\corref}[1]{Cor\-ol\-lary \ref{#1}}
\newcommand{\figref}[1]{Fig\-ure \ref{#1}}
\newcommand{\conjref}[1]{Con\-jec\-ture \ref{#1}}
\newcommand{\remref}[1]{Re\-mark \ref{#1}}
\newcommand{\probref}[1]{Prob\-lem \ref{#1}}

% Divides, does not divide.
\providecommand{\divides}{\mid}
\providecommand{\ndivides}{\nmid}
% Normal subgroup or equal.
\providecommand{\normaleq}{\unlhd}
% Normal subgroup.
\providecommand{\normal}{\lhd}
\providecommand{\rnormal}{\rhd}
\providecommand{\union}{\cup}
\providecommand{\bigunion}{\bigcup}
\providecommand{\intersect}{\cap}
\providecommand{\bigintersect}{\bigcap}


%% Homotopism arrows.
\newcommand{\ddd}{\downarrow\downarrow\downarrow}
\newcommand{\ddu}{\downarrow\downarrow\uparrow}
\newcommand{\dde}{\downarrow\downarrow\|}
\newcommand{\dud}{\downarrow\uparrow\downarrow}
\newcommand{\duu}{\downarrow\uparrow\uparrow}
\newcommand{\due}{\downarrow\uparrow\|}
\newcommand{\ded}{\downarrow\|\downarrow}
\newcommand{\deu}{\downarrow\|\uparrow}
\newcommand{\dee}{\downarrow\|\|}
%%
\newcommand{\udd}{\uparrow\downarrow\downarrow}
\newcommand{\udu}{\uparrow\downarrow\uparrow}
\newcommand{\ude}{\uparrow\downarrow\|}
\newcommand{\uud}{\uparrow\uparrow\downarrow}
\newcommand{\uuu}{\uparrow\uparrow\uparrow}
\newcommand{\uue}{\uparrow\uparrow\|}
\newcommand{\ued}{\uparrow\|\downarrow}
\newcommand{\ueu}{\uparrow\|\uparrow}
\newcommand{\uee}{\uparrow\|\|}
%%
\newcommand{\edd}{\|\downarrow\downarrow}
\newcommand{\edu}{\|\downarrow\uparrow}
\newcommand{\ede}{\|\downarrow\|}
\newcommand{\eud}{\|\uparrow\downarrow}
\newcommand{\euu}{\|\uparrow\uparrow}
\newcommand{\eue}{\|\uparrow\|}
\newcommand{\eed}{\|\|\downarrow}
\newcommand{\eeu}{\|\|\uparrow}
\newcommand{\eee}{\|\|\|}

\newcommand{\cev}[1]{\reflectbox{\ensuremath{\vec{\reflectbox{\ensuremath{#1}}}}}}

%--Shortcuts--

%%%%%%%%%%%%%%%%%%%%%%%%%%%%%%%%%%%%%%%

\DeclareMathOperator{\Hol}{Hol}
\DeclareMathOperator{\chr}{char }
\DeclareMathOperator{\trace}{tr~}
\DeclareMathOperator{\rad}{rad }
\DeclareMathOperator{\torrad}{torrad }
\DeclareMathOperator{\Fun}{Fun }
\DeclareMathOperator{\Hom}{hom }
\DeclareMathOperator{\End}{End}
\DeclareMathOperator{\Nil}{Nil }
\DeclareMathOperator{\Ric}{Rich }
\DeclareMathOperator{\Scal}{Scal }
\DeclareMathOperator{\Sym}{Sym }
\DeclareMathOperator{\Alt}{Alt }
\DeclareMathOperator{\Her}{Her }
\DeclareMathOperator{\Adj}{Adj }
\DeclareMathOperator{\Der}{Der }
\DeclareMathOperator{\Spin}{Spin }
\DeclareMathOperator{\JSpin}{JSpin }
\DeclareMathOperator{\GL}{GL}
\DeclareMathOperator{\PGL}{PGL}
\DeclareMathOperator{\SL}{SL}
\DeclareMathOperator{\Sp}{Sp}
\DeclareMathOperator{\GO}{GO}
\DeclareMathOperator{\GU}{GU}
\DeclareMathOperator{\GF}{GF}
\DeclareMathOperator{\Gal}{Gal }
\DeclareMathOperator{\Gr}{Gr}

% Lie algebras.
\DeclareMathOperator{\gl}{\mathfrak{gl}}
%\DeclareMathOperator{\sl}{\mathfrak{sl}}
\DeclareMathOperator{\so}{\mathfrak{so}}
%\DeclareMathOperator{\sp}{\mathfrak{sp}}
\DeclareMathOperator{\Inn}{Inn}
\DeclareMathOperator{\Aut}{Aut}
\DeclareMathOperator{\Inv}{Inv}
\DeclareMathOperator{\Isom}{Isom}
\DeclareMathOperator{\Str}{Str}
\DeclareMathOperator{\Stab}{Stab}
\DeclareMathOperator{\memb}{memb}
\DeclareMathOperator{\Cent}{Cen}
\DeclareMathOperator{\im}{im }
\DeclareMathOperator{\Res}{Res }
\DeclareMathOperator{\Ann}{Ann }
\DeclareMathOperator{\Prob}{Pr}
\DeclareMathOperator{\rank}{rank }
\DeclareMathOperator{\Diag}{Diag }
\DeclareMathOperator{\gr}{gr}
\DeclareMathOperator{\lcm}{lcm}
\DeclareMathOperator{\disc}{disc}
\DeclareMathOperator{\Out}{Out}
\DeclareMathOperator{\out}{out}

\newcommand{\Comm}[1]{{\mathsf{Comm}\textrm{-}{#1}}}
\newcommand{\Set}{{\mathsf{Set}}}

\DeclareMathOperator{\ad}{ad}

\newcommand{\M}{\mathbb{M}}
\newcommand{\cond}[2]{\overset{#1}{{_{#1}#2}_{#1}}}

\newcommand{\Bi}{\mathsf{Bi} }
\newcommand{\Grp}{\mathsf{Grp} }

\newcommand{\lt}[1]{{#1}^{\Lsh}}
\newcommand{\rt}[1]{{#1}^{\Rsh}}
\newcommand{\lr}[1]{{#1}^{\Lsh\Rsh}}
\newcommand{\ct}[1]{{#1}^{\uparrow}}
\newcommand{\perpsum}{\perp} %% \obot is prefered and uses mathabx
\newcommand{\op}{\circ}

\newcommand{\CC}{\mathcal{C}}
\newcommand{\LL}[1]{\mathcal{L}_{#1}}
\newcommand{\RR}[1]{\mathcal{R}_{#1}}
\newcommand{\MM}[1]{\mathcal{M}_{#1}}
\newcommand{\LMR}{\mathcal{LMR}}
\newcommand{\sprod}{\Pi\,}

%\usepackage{ifsym} % \TriangleRight
\newcommand{\trip}[3]{{_{#2}^{#1}} #3}

\newcommand{\lversor}{\,\reflectbox{\ensuremath{\oslash}}\,}%\obackslash}
\newcommand{\rversor}{\oslash}

\newcommand{\bm}{*}
\newcommand{\bmto}{\rightarrowtail}

%  TUPLES===============
%\usepackage[T1]{fontenc}% http://ctan.org/pkg/fontenc
\usepackage{aeguill}
\newcommand{\mm}[1]{\langle #1\rangle}
\newcommand{\la}{\langle}
\newcommand{\ra}{\rangle}
\newcommand{\lga}{\textnormal{\guillemotleft}}
\newcommand{\rga}{\textnormal{\guillemotright}}
\makeatletter
\newsavebox{\@brx}
\newcommand{\lla}[1][]{\savebox{\@brx}{\(\m@th{#1\langle}\)}%
  \mathopen{\copy\@brx\kern-0.5\wd\@brx\usebox{\@brx}}}
\newcommand{\rra}[1][]{\savebox{\@brx}{\(\m@th{#1\rangle}\)}%
  \mathclose{\copy\@brx\kern-0.5\wd\@brx\usebox{\@brx}}}
\makeatother

\newcommand{\bbinom}[2]{\begin{bmatrix} #1\\ #2 \end{bmatrix}}

\newcommand{\lsa}{\sphericalangle}
\newcommand{\rsa}{\reflectbox{\ensuremath{\sphericalangle}}}
\newcommand{\llb}{\llbracket}
\newcommand{\rrb}{\rrbracket}
\newcommand{\llp}{\llparenthesis}
\newcommand{\rrp}{\rrparenthesis}

%\newcommand{\TSat}[3]{#1_{\vDash #2(#3)}}
%\newcommand{\PSat}[3]{{_{#1\vDash}{#2}}_{#3}}
%\newcommand{\OpSat}[3]{{_{#1\vDash #2}#3}}

\newcommand{\bra}[1]{\la #1|}
\newcommand{\ket}[1]{| #1\ra}
\newcommand{\comp}[1]{\bar{#1}}
\newcommand{\widecomp}[1]{\overline{#1}}


%========== NICE HEBREW CHARACTERS  =====================

\usepackage{cjhebrew}
\DeclareFontFamily{U}{rcjhbltx}{}
\DeclareFontShape{U}{rcjhbltx}{m}{n}{<->rcjhbltx}{}
\DeclareSymbolFont{hebrewletters}{U}{rcjhbltx}{m}{n}

% remove the definitions from amssymb
\let\aleph\relax\let\beth\relax
\let\gimel\relax\let\daleth\relax

\DeclareMathSymbol{\aleph}{\mathord}{hebrewletters}{39}
\DeclareMathSymbol{\beth}{\mathord}{hebrewletters}{98}\let\bet\beth
\DeclareMathSymbol{\gimel}{\mathord}{hebrewletters}{103}
\DeclareMathSymbol{\daleth}{\mathord}{hebrewletters}{100}\let\dalet\daleth

\DeclareMathSymbol{\lamed}{\mathord}{hebrewletters}{108}
\DeclareMathSymbol{\mem}{\mathord}{hebrewletters}{109}\let\mim\mem
\DeclareMathSymbol{\ayin}{\mathord}{hebrewletters}{96}
\DeclareMathSymbol{\tsadi}{\mathord}{hebrewletters}{118}
\DeclareMathSymbol{\qof}{\mathord}{hebrewletters}{114}
\DeclareMathSymbol{\shin}{\mathord}{hebrewletters}{152}
\DeclareMathSymbol{\waw}{\mathord}{hebrewletters}{119}
\DeclareMathSymbol{\vavv}{\mathord}{hebrewletters}{119}
\DeclareMathOperator{\vav}{\ensuremath{\vavv}\,}


%-----------------------------------------------------------------------------
\begin{document}

\frontmatter

\dominitoc
\maketitle
\tableofcontents

\mainmatter

\chapter{Introduction}



\chapter{Invariants for bilinear tensors}

The following intrinsics are specialized for tensors of valence 3, but equivalent intrinsics for general tensors are presented in the proceeding subsection.

\index{AdjointAlgebra}
\begin{intrinsics}
AdjointAlgebra(t) : TenSpcElt -> AlgMat
\end{intrinsics}

Returns the adjoint $*$-algebra of the given Hermitian bilinear map $t$, represented on $\End(U_2)$. 
This is using algorithms from {\sc StarAlge}, see the \texttt{AdjointAlgebra} intrinsic in \cite{BW:StarAlge}.
If the current version of {\sc StarAlge} is not attached, the default {\sc Magma} version will be used instead.

\begin{example}[AdjointAlge]

Given the context of \cite{BW:isometry}, we construct a tensor from a $p$-group and compute its adjoint algebra.
\begin{code}
> G := SmallGroup(3^7, 7000);
> t := pCentralTensor(G);
> t;
Tensor of valence 3, U2 x U1 >-> U0
U2 : Full Vector space of degree 4 over GF(3)
U1 : Full Vector space of degree 4 over GF(3)
U0 : Full Vector space of degree 3 over GF(3)
\end{code}

Unlike other intrinsics that compute invariants of tensors, \texttt{AdjointAlgebra} exploits the fact that $t$ is Hermitian so that the adjoint algebra is faithfully represented on $\End(U_2)=\End(U_1)$. 
\begin{code}
> A := AdjointAlgebra(t);
> A;
Matrix Algebra of degree 4 and dimension 4 with 4 generators over GF(3)
> A.1;
[1 0 0 0]
[0 0 0 0]
[0 0 0 0]
[0 0 0 1]
\end{code}

Because $A$ is constructed from algorithms in {\sc StarAlge}, we can apply other algorithms from that package specifically dealing with the involution on $A$.
See \textsc{StarAlge} \cite{BW:StarAlge} for descriptions of the intrinsics.
\begin{code}
> RecognizeStarAlgebra(A);
true
> SimpleParameters(A);
[ <"symplectic", 2, 3> ]
> Star(A);
Mapping from: AlgMat: A to AlgMat: A given by a rule [no inverse]
\end{code}
\end{example}

\index{LeftNucleus!bilinear} 
\begin{intrinsics}
LeftNucleus(t) : TenSpcElt -> AlgMat
    op : BoolElt : false
\end{intrinsics}

Returns the left nucleus of the bilinear map $t$ as a subalgebra of $\End(U_2)\times \End(U_0)$.
In previous versions of TensorSpace (and eMAGma), the left nucleus was returned as a subalgebra of $\End(U_2)^\op\times \End(U_0)^\op$.
To enable this, set the optional arugment \texttt{op} to \texttt{true}.

\index{MidNucleus!bilinear} 
\begin{intrinsics}
MidNucleus(t) : TenSpcElt -> AlgMat
\end{intrinsics}

Returns the mid nucleus of the bilinear map $t$ as a subalgebra of $\End(U_2)\times \End(U_1)^\op$.

\index{RightNucleus!bilinear}
\begin{intrinsics}
RightNucleus(t) : TenSpcElt -> AlgMat
\end{intrinsics}

Returns the right nucleus of the bilinear map $t$ as a subalgebra of $\End(U_1)\times \End(U_0)$.

\begin{example}[GoingNuclear]

We will verify a theorem from \cite{FMW:densors} and \cite{Wilson:LMR}: all the nuclei of a tensor embed into the derivation algebra. 
We construct the tensor given by $(3\times 4\times 5)$-matrix multiplcation.
\begin{code}
> K := Rationals();
> A := KMatrixSpace(K, 3, 4);
> B := KMatrixSpace(K, 4, 5);
> C := KMatrixSpace(K, 3, 5);
> F := func< x | x[1]*x[2] >;
> t := Tensor([A, B, C], F);
> t;
Tensor of valence 3, U2 x U1 >-> U0
U2 : Full Vector space of degree 12 over Rational Field
U1 : Full Vector space of degree 20 over Rational Field
U0 : Full Vector space of degree 15 over Rational Field
\end{code}

Because matrix multiplication is associative, the left, middle, and right nuclei contain $\mathbb{M}_3(\mathbb{Q})$, $\mathbb{M}_4(\mathbb{Q})$, and $\mathbb{M}_5(\mathbb{Q})$ respectively.
In fact, the following computation shows that this is equality.
\begin{code}
> L := LeftNucleus(t : op := true);
> M := MidNucleus(t);
> R := RightNucleus(t);
> Dimension(L), Dimension(M), Dimension(R);
9 16 25
> D := DerivationAlgebra(t);
> Dimension(D);
49
\end{code}

Now we will embed these nuclei into the derivation algebra of $t$. 
\begin{code}
> Omega := KMatrixSpace(K, 47, 47);
> Z1 := ZeroMatrix(K, 20, 20);
> L_L2, L2 := Induce(L, 2);
> L_L0, L0 := Induce(L, 0);
> embedL := map< L -> Omega | x :-> 
>     DiagonalJoin(<Transpose(x @ L_L2), Z1, Transpose(x @ L_L0)>) >;
> 
> Z0 := ZeroMatrix(K, 15, 15);
> M_M2, M2 := Induce(M, 2);
> M_M1, M1 := Induce(M, 1);
> embedM := map< M -> Omega | x :->
>     DiagonalJoin(<x @ M_M2, -Transpose(x @ M_M1), Z0>) >;
> 
> Z2 := ZeroMatrix(K, 12, 12);
> R_R1, R1 := Induce(R, 1);
> R_R0, R0 := Induce(R, 0);
> embedR := map< R -> Omega | x :->
>     DiagonalJoin(<Z2, x @ R_R1, x @ R_R0>) >;
> 
> Random(Basis(L)) @ embedL in D;
true
> Random(Basis(M)) @ embedM in D;
true
> Random(Basis(R)) @ embedR in D;
true
\end{code}
\end{example}



\chapter{Invariants of general multilinear maps}

The following functions can be used for general tensors.

\index{Centroid!tensor}
\begin{intrinsics}
Centroid(t) : TenSpcElt -> AlgMat
Centroid(t, A) : TenSpcElt, {RngIntElt} -> AlgMat
\end{intrinsics}

Returns the $A$-centroid of the tensor as a subalgebra of $\prod_{a\in A}\End(U_a)$, where $A\subseteq [\vav]$.
If no $A$ is given, it is assumed that $A=[\vav]$. 
If $t$ is contained in a category where coordinates $a$ and $b$ (also contained in $A$) are fused together, then the corresponding operators on those coordinates will be equal. 

\begin{example}[Centroid]

The centroid $C$ of a tensor $t$ is the largest ring for which $t$ is $C$-linear, see \cite{FMW:densors}*{Theorem~D}. 
To demonstrate this, we will construct the tensor given by multiplication of the splitting field of $f(x)=x^4-x^2-2$ over $\mathbb{Q}$.
However, this field won't explicitly be given with the tensor data.
\begin{code}
> A := MatrixAlgebra(Rationals(), 4);
> R<x> := PolynomialRing(Rationals());
> F := sub< A | A!1, CompanionMatrix(x^4-x^2-2) >;
> F;
Matrix Algebra of degree 4 with 2 generators over Rational Field
> t := Tensor(F);
> t;
Tensor of valence 3, U2 x U1 >-> U0
U2 : Full Vector space of degree 4 over Rational Field
U1 : Full Vector space of degree 4 over Rational Field
U0 : Full Vector space of degree 4 over Rational Field
\end{code}

The centroid is the field $\mathbb{Q}(\sqrt{2},i)$. 
\begin{code}
> C := Centroid(t);
> C;
Matrix Algebra of degree 12 with 4 generators over Rational Field
> sub< C | C.1 > eq C;
true
> forall{ c : c in Generators(C) | IsInvertible(c) };
true
> IsCommutative(C);
true
> MinimalPolynomial(C.1);
x^4 + 2*x^2 - 8
> Factorization(MinimalPolynomial(C.1));
[
    <x^2 - 2, 1>,
    <x^2 + 4, 1>
]
\end{code}
\end{example}


\index{DerivationAlgebra!tensor}
\begin{intrinsics}
DerivationAlgebra(t) : TenSpcElt -> AlgMatLie
DerivationAlgebra(t, A) : TenSpcElt, {RngIntElt} -> AlgMatLie
\end{intrinsics}

Returns the $A$-derivation Lie algebra of the tensor as a Lie subalgebra of $\prod_{a\in A}\End(U_a)$, where $A\subseteq [\vav]$.
If no $A$ is given, it is assumed that $A=[\vav]$. 
If $t$ is contained in a category where coordinates $a$ and $b$ (also contained in $A$) are fused together, then the corresponding operators on those coordinates will be equal.

\index{Nucleus}
\begin{intrinsics}
Nucleus(t, a, b) : TenSpcElt, RngIntElt, RngIntElt -> AlgMat
Nucleus(t, A) : TenSpcElt, SetEnum -> AlgMat
\end{intrinsics}

Returns the $A$-nucleus, for $A=\{a,b\}$ ($a\ne b$), of the tensor as a subalgebra of $\End(U_i)\times \End(U_j)$, 
where $i=\max(a,b)$ and $j=\min(a,b)$.
If $j>0$, then replace $\End(U_j)$ with $\End(U_j)^\op$. 
If $t$ is contained in a category where coordinates $a$ and $b$ are fused together, then the corresponding operators on those coordinates will not be forced to be equal.

\begin{example}[RestrictDerivation]

In a previous example, we embeded the nuclei of a tensor into the derivation algebra. 
For a tensor $t: U_{\vav}\times \cdots\times U_1\rightarrowtail U_0$, the derivation algebra is represented in $\Omega=\prod_{a\in[\vav]}\mathfrak{gl}(U_a)$.
We will restrict the derivation algebra to $\prod_{c\notin\{a,b\}} \mathfrak{gl}(U_c)$ for distinct $a,b\in[\vav]$. 
From \cite{FMW:densors}*{Lemma~4.11}, the kernel of this restriction is equal to Nuc$_{\{a,b\}}(t)^-$.
We will just verify that the dimensions match.

We will construct a tensor given by matrix multiplcation: 
\[ \mathbb{M}_{3\times 4}(\mathbb{F}_2)\times \mathbb{M}_{4\times 2}(\mathbb{F}_2) \times \mathbb{M}_{2\times 2}(\mathbb{F}_2) \rightarrowtail \mathbb{M}_{3\times 2}(\mathbb{F}_2) .\]
\begin{code}
> A := KMatrixSpace(GF(2), 3, 4);
> B := KMatrixSpace(GF(2), 4, 2);
> C := KMatrixSpace(GF(2), 2, 2);
> D := KMatrixSpace(GF(2), 3, 2);
> trip := func< x | x[1]*x[2]*x[3] >;
> t := Tensor([A, B, C, D], trip);
> t;
Tensor of valence 4, U3 x U2 x U1 >-> U0
U3 : Full Vector space of degree 12 over GF(2)
U2 : Full Vector space of degree 8 over GF(2)
U1 : Full Vector space of degree 4 over GF(2)
U0 : Full Vector space of degree 6 over GF(2)
\end{code}

Now we will compute the derivation algebra of \texttt{t}. 
We choose $a=3$ and $b=2$, so the $\{3,2\}$-nucleus is $\mathbb{M}_{4\times 4}(\mathbb{F}_2)$.
\begin{code}
> D := DerivationAlgebra(t);
> Dimension(D);
32
> N32 := Nucleus(t, 3, 2);
> N32;
Matrix Algebra of degree 20 with 16 generators over GF(2)
\end{code}

To construct the restriction of $\Der(t)$ into $\mathfrak{gl}(U_1)\times\mathfrak{gl}(U_0)$ we will use the \texttt{Induce} function. 
\begin{code}
> Omega_10 := KMatrixSpace(GF(2), 10, 10);
> D_vs := sub< KMatrixSpace(GF(2), 30, 30) | Basis(D) >;
> pi1, D1 := Induce(D, 1);
> pi0, D0 := Induce(D, 0);
> res := hom< D_vs -> Omega_10 | 
>     [<x, DiagonalJoin(x @ pi1, x @ pi0)> : x in Basis(D)] >;
> res;
Mapping from: ModMatFld: D_vs to ModMatFld: Omega_10
> Kernel(res);
KMatrixSpace of 30 by 30 matrices and dimension 16 over GF(2)
\end{code}
\end{example}

\index{SelfAdjointAlgebra}
\begin{intrinsics}
SelfAdjointAlgebra(t, a, b) : TenSpcElt, RngIntElt, RngIntElt -> ModMatFld
\end{intrinsics}

Returns the self-adjoint elements of the $ab$-nucleus of $t$ as a subspace of $\End(U_a)$, with $a,b\in[\vav]$ and $a\ne b$. 
It is not required that $t$ be in a tensor category with coordinates $a$ and $b$ fused. 
Unlike other invariants associated to tensors, the self-adjoint algebra is not currently stored with the tensor.

\index{TensorOverCentroid}
\begin{intrinsics}
TensorOverCentroid(t) : TenSpcElt -> TenSpcElt, Hmtp
\end{intrinsics}

If the given tensor $t$ is framed by $K$-vector spaces, then the returned tensor is framed by $E$-vector spaces where $E$ is the residue field of the centroid. 
The returned homotopism is an isotopism of the $K$-tensors.
This only works if the centroid of $t$ is a finite commutative local ring.
We employ the algorithms developed by Brooksbank and Wilson \cite{BW:Module-iso} to efficiently determine if a matrix algebra is cyclic, see Appendix~\ref{append:cyclic}.

\begin{example}[CentroidUnipotent]

In the context of groups, centroids can be used to recover an underlying field of a matrix group, even if the given group is not input as such.
Here we will construct the exponent-$p$ central tensor of the Sylow 2-subgroup of $\GL(3,\GF(2^{10}))$. 
We will not print the \texttt{GrpPC} version of this group as the number of relations is very large.
\begin{code}
> U := ClassicalSylow(GL(3, 2^10), 2);
> U.3;
[       1    $.1^2        0]
[       0        1        0]
[       0        0        1]
> G := PCPresentation(UnipotentMatrixGroup(U));
> #G eq 2^30;
true
> t := pCentralTensor(G);
> t;
Tensor of valence 3, U2 x U1 >-> U0
U2 : Full Vector space of degree 20 over GF(2)
U1 : Full Vector space of degree 20 over GF(2)
U0 : Full Vector space of degree 10 over GF(2)
\end{code} 

Even though our tensor right now is $\mathbb{F}_2^{20}\times \mathbb{F}_2^{20}\rightarrowtail\mathbb{F}_2^{10}$, we know it is the 2-dimensional alternating form over $\GF(2^{10})$. 
We will construct the centroid, and then rewrite our tensor over the centroid to get the tensor we expect.
\begin{code}
> C := Centroid(t);
> C;
Matrix Algebra of degree 50 and dimension 10 with 1 generator over GF(2)
> IsCyclic(C) and IsSimple(C);
true
> s := TensorOverCentroid(t);
> s;
Tensor of valence 3, U2 x U1 >-> U0
U2 : Full Vector space of degree 2 over GF(2^10)
U1 : Full Vector space of degree 2 over GF(2^10)
U0 : Full Vector space of degree 1 over GF(2^10)
\end{code}
\end{example}




\backmatter

\begin{bibdiv}
\begin{biblist}



\end{biblist}
\end{bibdiv}

\printindex


\end{document}
